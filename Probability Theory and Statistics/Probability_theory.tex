% !TXS template
\documentclass[french]{article}

\usepackage[utf8]{inputenc}
\usepackage[bulgarian]{babel}
\usepackage{amsmath}
\usepackage{amssymb}
\usepackage{bigints}
\usepackage{graphicx}
\usepackage{wrapfig}
\usepackage{relsize}
\graphicspath{ {"C:/Users/Maria/Desktop/Probability and Statistics/Homework/Task_04.png"} }
\newcommand{\probP}{\text{I\kern-0.15em P}}
% Title Page

\title{Вероятности и Статистика}
\author{Мария Гроздева}
\date{2022г.}


\begin{document}
	\maketitle
	\author
	
	\subsection*{Задача 1:}
	От $10$ стандартни тестета от $52$ карти се тегли по една карта. Намерете вероятността в получената ръка от $10$ карти:
	\begin{itemize}
		\item да няма повтарящи се;
		\item да има поне три аса;
		\item да има четири спатии, три кари, две купи и една пика;
		\item броят на черните карти да е с точно $4$ повече от броя на червените, ако е известно, че черните карти са повече от червените.
	\end{itemize}

	\subsection*{Решение:}
	\begin{itemize}
		\item $A = \{\text{в получената ръка от $10$ карти няма повтарящи се}\}$ \\
		$\probP(A) = \dfrac{52*51*50*49*48*47*46*45*44*43}{52^{10}} \approx 0.3971$
		
		\item $B = \{\text{в получената ръка от $10$ карти има поне три аса}\}$ \\
		$B_0 = \{\text{в получената ръка от $10$ карти има точно 0 аса}\}$ \\
		$B_1 = \{\text{в получената ръка от $10$ карти има точно 1 асо}\}$ \\
		$B_2 = \{\text{в получената ръка от $10$ карти има точно 2 аса}\}$ \\
		\\
		$\probP(B_0) = \dfrac{48^{10}}{52^{10}} \approx 0.4491$ \\
		$\probP(B_1) = \dfrac{\dbinom{10}{1}\dbinom{4}{1}48^9}{52^{10}} \approx 0.3742$ \\
		$\probP(B_2) = \dfrac{\dbinom{10}{2}\dbinom{4}{1}\dbinom{4}{1}48^8}{52^{10}} \approx 0.1403$ \\
		\\ \\
		$\probP(B) = 1 - \probP(B_0 \cup B_1 \cup B_2) \overset{\mathrm{\text{несъвместими}}}{=} 1 - (\probP(B_0) + \probP(B_1) + \probP(B_2)) \approx \\ \approx 1 - (0.4491 + 0.3742 + 0.1403) = 1 - 0.9636 = 0.0364$
		
		\item $C = \{\text{в получената ръка от $10$ карти има четири спатии, три кари, две купи и една пика}\}$ \\
		$\probP({C}) = \dfrac{\dbinom{10}{4}13^4\dbinom{6}{3}13^3\dbinom{3}{2}13^2\dbinom{1}{1}13^1}{52^{10}} = \dfrac{1575}{131072} \approx 0.0120$
		
		\item $D = \{$в получената ръка от $10$ карти броят на черните карти е с точно 4 повече от броя на червените, ако е известно, че черните карти са повече от червените$\}$ \\
		$D_1 = \{$в получената ръка от $10$ карти броят на черните карти е с точно 4 повече от броя на червените$\}$ \\
		$D_2 = \{$в получената ръка от $10$ карти черните карти са повече от червените$\}$ \\
		\\
		$\probP({D}) = \probP({D_1} | {D_2}) = \dfrac{\probP({D_1} \cap {D_2})}{\probP({D_2})} = \dfrac{15}{128} * \dfrac{512}{193} = \dfrac{60}{193} \approx 0.311$ \\
		\\ \\
		$\probP({D_1} \cap {D_2}) = \dfrac{26^7*26^3*\dbinom{10}{7}}{52^{10}} = \dfrac{15}{128}$ \\
		\\ \\
		$\probP({D_2}) = \dfrac{26^{10}*\dbinom{10}{6} + 26^{10}*\dbinom{10}{7} + 26^{10}*\dbinom{10}{8} + 26^{10}*\dbinom{10}{9} + 26^{10}*\dbinom{10}{10}}{52^{10}} = \\ = \dfrac{\sum_{k=6}^{10}\dbinom{10}{k}}{2^{10}} = \dfrac{193}{512}$
	\end{itemize}


	\subsection*{Задача 2:}
	Всички изделия в дадена партида са изправни, а в друга - $1/4$ от изделията са за брак. Изделие, взето от случайно избрана партида, се оказва изправно. Да се пресметне вероятността второ случайно избрано изделие от същата партида да се окаже за брак, ако след проверката на първото изделие то е било върнато обратно в своята партида.
	
	\subsection*{Решение:}
	$H_1 = \{\text{първото изделие е взето от първата партида}\}$ \\
	$H_2 = \{\text{първото изделие е взето от втората партида}\}$ \\
	$\probP(H_1) = \probP(H_2) = \dfrac{1}{2}$ \\
	\\
	$A = \{\text{първото изделие е изправно}\}$ \\
	$\probP(A) \overset{\mathrm{\text{пълна вероятност}}}{=} \probP(A | H_1)*\probP(H_1)\ +\ \probP(A | H_2)*\probP(H_2) = 1*\dfrac{1}{2} + \dfrac{3}{4}*\dfrac{1}{2} = \dfrac{7}{8}$ \\
	\\ \\
	$\probP(H_1|A) \overset{\mathrm{\text{ф-ла на Бейс}}}{=} \dfrac{\probP(A|H_1)*\probP(H_1)}{\probP(A)} = \dfrac{1/2}{7/8} = \dfrac{4}{7}$ \\
	$\probP(H_2|A) \overset{\mathrm{\text{ф-ла на Бейс}}}{=} \dfrac{\probP(A|H_2)*\probP(H_2)}{\probP(A)} = \dfrac{3/8}{7/8} = \dfrac{3}{7}$ \\
	($H_1$ има по-голяма апостериорна вероятност от $H_2$.) \\
	\\
	$B = \{\text{второто изделие е за брак}\}$ \\
	$\probP(B) \overset{\mathrm{\text{пълна вероятност}}}{=} \probP(B | H_1)*\probP(H_1)\ +\ \probP(B | H_2)*\probP(H_2) = 0 + \dfrac{1}{4}*\dfrac{3}{7} = \dfrac{3}{28}$
	
	
	\subsection*{Задача 3:}
	Дете има в левия си джоб четири монети от $1$лв. и три монети от $2$лв., а в десния си джоб - две монети от $1$лв. и една монета от $2$лв. Детето прехвърля две монети от левия си джоб в десния, след това връща обратно две монети от десния джоб в левия. Накрая детето вади монета от десния си джоб. Каква е вероятността тя да е от $1$лв.?
	
	\subsection*{Решение:}
	$A = \{\text{детето вади монета от $1$лв. от десния си джоб}\}$ \\
	\\
	$H_{l11} = \{\text{детето прехвърля две монети от $1$лв. от левия си джоб в десния}\}$ \\
	$\probP(H_{l11}) = \dfrac{4}{7}*\dfrac{3}{6} = \dfrac{2}{7}$ \\
	\\
	$H_{l22} = \{\text{детето прехвърля две монети от $2$лв. от левия си джоб в десния}\}$ \\
	$\probP(H_{l22}) = \dfrac{3}{7}*\dfrac{2}{6} = \dfrac{1}{7}$ \\
	\\
	$H_{l12} = \{\text{детето прехвърля една монета от $1$лв. и една монета от $2$лв. от левия си джоб в десния}\}$ \\
	$\probP(H_{l12}) = \dfrac{4}{7}*\dfrac{3}{6} + \dfrac{3}{7}*\dfrac{4}{6} = \dfrac{4}{7}$ \\
	\\ \\
	$\probP(A) \overset{\mathrm{\text{пълна вероятност}}}{=} \probP(A | H_{l11})*\probP(H_{l11}) + \probP(A | H_{l22})*\probP(H_{l22}) + \probP(A | H_{l12})*\probP(H_{l12}) = \dfrac{4}{5}*\dfrac{2}{7} + \dfrac{2}{5}*\dfrac{1}{7} + \dfrac{3}{5}*\dfrac{4}{7} = \dfrac{22}{35} \approx 0.629$ \\
	\\
	\textbf{Забележка:} Второто преместване на две монети от \textbf{десния към левия} джоб \textbf{НЕ} оказва влияние върху търсената вероятност. \\ \\
	Можем да мислим преместването на две монети (от десен към ляв джоб) и изваждането на трета като едно действие по следния начин: вадим три монети от десния джоб - първите две прехвърляме в левия, третата - изваждаме. Тези три монети, от своя страна, можем да мислим като наредена тройка. Така свеждаме търсената вероятност до вероятността на трета позиция в наредената тройка (монетата, която вадим) да се окаже монета от $1$лв. Както виждаме, тази вероятност по никакъв начин не се влияе от преместването на другите две монети. \\
	До същия извод можем да стигнем, ако приложим формулата за пълна вероятност за всички случаи (общо 9).
	
	
	\subsection*{Задача 4:}
	Каква е вероятността корените на квадратното уравнение \\ $x^2 + ax + b = 0, a, b \in [0, 1]$ да бъдат реални числа?
	
	\subsection*{Решение:}
	$A = \{\text{корените на квадратното уравнение са реални числа}\}$ \\
	\begin{wrapfigure}{r}{0.25\textwidth}
		\includegraphics[width=1.5in, height=1.5in]{Task_04.png}
		\label{fig}
	\end{wrapfigure}
	$x^2 + ax + b = 0$ \\
	$x_{1,2} = \dfrac{-a \pm \sqrt{a^2 - 4b}}{2}$ 
	\\
	Следователно корените на уравнението са реални числа $\Longleftrightarrow a^2 \ge 4b \Longleftrightarrow a^2/4 \ge b$. \\ \\
	Следователно $A = \{(a,b)\ |\ a^2/4 \ge b\}$; \\
	$\Omega = \{(a,b)\ |\  a, b \in [0, 1]\}$ \\
	\\
	$\probP(A) = \dfrac{S_A}{S_\Omega} = \bigintss_{0}^{1}\dfrac{a^2}{4}da = \dfrac{a^3}{12} \Big|_0^1 = \dfrac{1}{12}$
	
	
	\subsection*{Задача 5:}
	Два различими зара се хвърлят един след друг последователно десет пъти. Каква е вероятността броят на хвърлянията, при които на първия зар се падат повече точки, отколкото на втория, да бъде:
	\begin{itemize}
		\item точно 4;
		\item не повече от 5?
	\end{itemize}
	
	\subsection*{Решение:}
	$A = \{\text{на първия зар се падат повече точки, отколкото на втория}\}$ \\
	$\probP(A) = \dfrac{15}{36}$ \\
	\\
	Нека случайната величина $X$ е броят на хвърлянията, при които на първия зар се падат повече точки, отколкото на втория. \\
	$X \sim Bi\left(10, \dfrac{15}{36}\right)$ \\
	\begin{itemize}
		\item $\probP(X = 4) = \dbinom{10}{4}*\left(\dfrac{15}{36}\right)^4*\left(\dfrac{21}{36}\right)^6 \approx 0.249$
		\item $\probP(X \le 5) = \probP(X = 0 \cup X = 1 \cup X = 2 \cup X = 3 \cup X = 4 \cup X = 5) = \\ \overset{\mathrm{\text{несъвместими}}}{=} \probP(X = 0) + \probP(X = 1) + \probP(X = 2) + \probP(X = 3) + \\ + \probP(X = 4) + \probP(X = 5) \approx 0.8046$
	\end{itemize}
	
	
	\subsection*{Задача 6:}
	Две от страните на правилен зар са оцветени в бяло, други две - в зелено, последните две - в червено. Хвърляме този зар два пъти. Нека $X$ е броят на падналите се бели, а $Y$ - на падналите се червени страни. Да се намерят: съвместното разпределение на $X$ и $Y$, независими ли са, ковариацията им, $\probP(X = 1\ |\ Y = 1)$ и $\probP(X > Y)$.
	
	\subsection*{Решение:}
	\begin{enumerate}
		\item Съвместно разпределение на $X$ и $Y$: \\ \\
		\begin{tabular}{||c c c c||} 
			\hline
			$X/Y$ & $0$ & $1$ & $2$ \\ [0.5ex] 
			\hline\hline
			$0$ & $1/9$ & $2/9$ & $1/9$ \\ 
			\hline
			$1$ & $2/9$ & $2/9$ & $0$ \\
			\hline
			$2$ & $1/9$ & $0$ & $0$ \\
			\hline
		\end{tabular}
	
		\item $X \not\perp \!\!\! \perp Y$. \\
		Например: $0 = \probP(X = 2 \cap Y = 2) \not= \probP(X = 2)\probP(Y = 2) = \dfrac{1}{9}*\dfrac{1}{9} = \dfrac{1}{81}$
		
		\item $\mathbb{E}X = \mathbb{E}Y = 0*\dfrac{4}{9} + 1*\dfrac{4}{9} + 2*\dfrac{1}{9} = \dfrac{2}{3} \approx 0.6667$ \\ \\
		$\mathbb{E}XY = 1*1*\dfrac{2}{9} = \dfrac{2}{9}$ \\
		\\
		$cov(X, Y) = \mathbb{E}XY - \mathbb{E}X\mathbb{E}Y = \dfrac{2}{9} - \dfrac{2}{3}*\dfrac{2}{3} = \dfrac{2}{9} - \dfrac{4}{9} = -\dfrac{2}{9}$
		
		\item $\probP(X = 1\ |\ Y = 1) = \dfrac{\probP(X = 1 \cap Y = 1)}{\probP(Y = 1)} = \dfrac{2/9}{4/9} = \dfrac{1}{2}$
		
		\item $\probP(X > Y) = \\
		\probP((X = 1\ \cap\ Y = 0)\ \cup\ (X = 2\ \cap\ Y = 0)\ \cup\ (X = 2\ \cap\ Y = 1)) \overset{\mathrm{\text{несъвместими}}}{=} \\
		\probP(X = 1\ \cap\ Y = 0) + \probP(X = 2\ \cap\ Y = 0) + \probP(X = 2\ \cap\ Y = 1) = \\
		\dfrac{2}{9} + \dfrac{1}{9} = \dfrac{1}{3}$
	\end{enumerate}


	\subsection*{Задача 7:}
	Нека сл. вел. $X$ приема стойности в $\mathbb{N}$ и
	$$g_X(s) := \mathbb{E}s^X = \sum_{k=0}^{\infty}s^k\probP(X=k)$$
	е пораждащата й функция.
	\begin{enumerate}
		\item Нека $Y := 3X$ и $Z := X_1 + X_2$, където $X_1, X_2 \sim X$ са независими. Изразете чрез $g_X$ пораждащите функции $g_Y$ и $g_Z$.
		\item Нека $X \sim Ge(p)$, т.е. $\probP(X=k) = p(1-p)^k$ за $k \in \mathbb{N}$. Пресметнете $g_X$ и чрез нейна помощ намерете $\mathbb{E}X$ и $DX$.
	\end{enumerate}

	\subsection*{Решение:}
	\begin{enumerate}
		\item $g_Y(s) = \mathbb{E}s^Y \overset{\mathrm{\text{деф. $Y$}}}{=} \mathbb{E}s^{3X} = \mathbb{E}(s^3)^X = g_X(s^3) = \mathlarger{\sum_{k=0}^{\infty}s^{3k}\probP(X=k)}$ \\
		$g_Z(s) = g_{X_1+X_2}(s) \overset{\mathrm{\text{$X_1 \perp \!\!\! \perp X_2$}}}{=} g_{X_1}(s)g_{X_2}(s) \overset{\mathrm{\text{$X_1, X_2 \sim X$}}}{=} g_X(s)g_X(s) = (g_X(s))^2$ \\
		
		\item Нека $q := 1-p$. \\
		$g_X(s) = \mathbb{E}s^X = \sum_{k=0}^{\infty}s^k\probP(X=k) = \sum_{k=0}^{\infty}s^kq^kp = \dfrac{q}{1-qs}$ \\
		$g'_X(s) = \dfrac{pq}{(1-qs)^2}$ \\
		$g''_X(s) = \dfrac{2pq^2(1-qs)}{(1-qs)^4}$ \\ \\
		\\
		$\mathbb{E}X = g'_X(1) = \dfrac{q}{p}$. \\
		$DX = g'_X(1) + g''_X(1) - (g'_X(1))^2 = \dfrac{q}{p} + \dfrac{2q^2}{p^2} - \dfrac{q^2}{p^2} = \dfrac{q(p+q)}{p^2} = \dfrac{q}{p^2}$.
	\end{enumerate}


	\subsection*{Задача 8:}
	Нека $X$ и $Y$ са независими случайни величини и
	$$\probP(X = -1) = \probP(X = 1) = \probP(Y = 1) = 2\probP(Y = 3) = 2\probP(Y = 5) = \dfrac{1}{2}.$$
	Нека $Z_1 := 2X + Y + 1, Z_2 = XY$. Намерете очакванията и дисперсиите им.
	
	\subsection*{Решение:}
	\begin{tabular}{||c c c||} 
		\hline
		$X$ & $-1$ & $1$ \\ [0.5ex] 
		\hline\hline
		$\probP$ & $1/2$ & $1/2$ \\ 
		\hline
	\end{tabular}
	\begin{tabular}{||c c c c||} 
	\hline
	$Y$ & $1$ & $3$ & $5$ \\ [0.5ex] 
	\hline\hline
	$\probP$ & $1/2$ & $1/4$ & $1/4$ \\ 
	\hline
	\end{tabular}
	\\ \\ \\
	$\mathbb{E}X = -\dfrac{1}{2} + \dfrac{1}{2} = 0$. \\
	$\mathbb{E}Y = \dfrac{1}{2} + \dfrac{3}{4} + \dfrac{5}{4} = \dfrac{5}{2}$. \\
	\\
	$DX =\mathbb{E}X^2 - (\mathbb{E}X)^2 = 1$. \\
	$DY =\mathbb{E}Y^2 - (\mathbb{E}Y)^2 = 9 - \dfrac{25}{4} = \dfrac{11}{4}$. \\
	\\ \\
	$\mathbb{E}Z_1 = \mathbb{E}(2X + Y + 1) = \mathbb{E}2X + \mathbb{E}Y + \mathbb{E}1 = 2\mathbb{E}X + \mathbb{E}Y + 1 = \dfrac{7}{2}$. \\
	$DZ_1 = D(2X + Y + 1) \overset{\mathrm{\text{$\perp \!\!\! \perp$}}}{=} D2X + DY + D1 = 4DX + DY + 0 = 4 + \dfrac{11}{4} = \dfrac{27}{4}$. \\
	\\ \\
	$\mathbb{E}Z_2 = \mathbb{E}XY = \sum_{i=1}^{2}\sum_{j=1}^{3}x_iy_j\probP(X = x_i \cap Y = y_j) \overset{\mathrm{\text{$\perp \!\!\! \perp$}}}{=} \\ = \sum_{i=1}^{2}\sum_{j=1}^{3}x_iy_j\probP(X = x_i)\probP(Y = y_j) = -\dfrac{1}{4} - \dfrac{3}{8} - \dfrac{5}{8} + \dfrac{1}{4} + \dfrac{3}{8} + \dfrac{5}{8} = 0$. \\ \\
	$DZ_2 = DXY = \mathbb{E}(XY)^2 - (\mathbb{E}XY)^2 = \dfrac{1}{4} + \dfrac{9}{8} + \dfrac{25}{8} + \dfrac{1}{4} + \dfrac{9}{8} + \dfrac{25}{8} = 9$. \\


	\subsection*{Задача 9:}
	Човек се намира на числовата ос в точката $n \in \mathbb{N}$ и последователно прави стъпка към $(n + 1)$ с	вероятност $p > 1/2$ и към $(n - 1)$ с вероятност $(1 - p)$. \\
	Нека $p_n = \probP$("Човекът достига $0$, тръгвайки от $n$"). \\
	Изразете $p_1,$ $p_2$ и $p_3$ чрез $p$. \\
	(Можете ли да съобразите, че $p_2 = p_1^k$ за някакво $k$? Колко е $\lim_{n\to\infty}p_n$? А ако $p < 1/2$?)
	
	\subsection*{Решение:}
	$p_1 = \probP$("Човекът достига $0$, тръгвайки от $1$"). \\
	Но вероятностите за преход $p$ и $1 - p$ са еднакви във всички точки. \\
	Затова $p_1 = \probP$("Човекът достига $n-1$, тръгвайки от $n$"). \\
	\\
	$p_2 = \probP$("Човекът достига $0$, тръгвайки от $2$") = \\
	= $\probP$("Човекът достига $1$, тръгвайки от $2$")*$\probP$("Човекът достига $0$, тръгвайки от $1$") = $p_1*p_1 = p_1^2$. \\
	\\
	Аналогично $p_3 = p_1^3$ и изобщо $p_k = p_1^k$. \\
	\\
	По формулата за пълна вероятност: \\
	$p_1 = \probP$("Човекът достига $0$, тръгвайки от $1$") = \\ = $\probP$(\{Първата стъпка да е към $0$\})*$1$ + $\probP$(\{Първата стъпка да е към $2$\})*$p_2 = (1-p) + p*p_2 = (1-p) + p*p_1^2$. \\
	\\
	Получихме квадратното уравнение относно $p_1$: \\
	$p_1 = (1 - p) + p * p_1^2$, тоест \\
	$p * p_1^2 - p_1 + (1 - p) = 0$. \\ \\
	Като го решим, намираме \\
	$p_1 = \dfrac{1 \pm \sqrt{1 - 4(1 - p)p}}{2p} \\
	p_1 = \dfrac{1 \pm 2p - 1}{2p} \\
	p_1 = 1$ или $p_1 = \dfrac{1 - p}{p}$. \\
	\\
	Следователно $p_1 = \dfrac{1 - p}{p}$. \\
	\\
	От изведеното в твърдение следва, че $p_k = p_1^k = \left(\dfrac{1 - p}{p}\right)^k$ за $k = 1,2,...$ \\
	\\
	Щом $0 < p_1 < 1$, то $\lim_{n\to\infty}p_n = \lim_{n\to\infty}p_1^n = 0$.
	
	
	\subsection*{Задача 10:}
	Студенти влизат последователно на изпит, показвайки личната си карта. Преди изпита е обявено, че първият студент, чийто рожден ден съвпада с рождения ден на вече влязъл студент, ще получи единица бонус към оценката си. На кое място трябва да застанете в редицата от студенти, за да имате най-голям шанс да сте печелившия студент?
	
	\subsection*{Решение:}
	$\probP(\{$Вероятността $(n + 1)$-вият влязъл студент да е първият, чийто рожден ден съвпада с някой от предишните $n\}) = \dfrac{364}{365}*\dfrac{363}{365}*\dfrac{362}{365}*...*\dfrac{365 - n}{365}*\dfrac{n}{365} = \\ = \left(1 - \dfrac{1}{365}\right)\left(1 - \dfrac{2}{365}\right)\left(1 - \dfrac{3}{365}\right)...\left(1 - \dfrac{n}{365}\right)*\dfrac{n}{365}$ \\
	\\
	Нека означим получения израз с $\probP(n)$. \\
	Търсим $n$, за което $\probP(n)$ = $max$.
	\\
	За целта разглеждаме частното $\dfrac{\probP(n)}{\probP(n-1)} = \dfrac{\left(1 - \dfrac{n}{365}\right)*\dfrac{n}{365}}{\dfrac{n-1}{365}} = \\ = \left(1 - \dfrac{n}{365}\right)*\dfrac{n}{n-1}$ \\
	\\
	Сравняваме това частно с $1$: \\
	$\left(1 - \dfrac{n}{365}\right)*\dfrac{n}{n-1} > 1 \Longleftrightarrow \\
	1 - \dfrac{n}{365} > \dfrac{n}{n-1} \Longleftrightarrow \\
	1 - \dfrac{n}{365} > 1 - \dfrac{1}{n} \Longleftrightarrow \\
	\dfrac{n}{365} < \dfrac{1}{n} \Longleftrightarrow \\
	n^2 < 365 \Longleftrightarrow \\
	n < \sqrt{365} \Longleftrightarrow \\
	n \le 19.$ \\
	\\
	Значи $\probP(1) < \probP(2) < \probP(3) < ... < \probP(18) < \probP(19) > \probP(20) > \probP(21) > ...$ \\
	Следователно най-голямата вероятност е $\probP(19)$. \\
	Тоест $n = 19$, откъдето $n + 1 = 20$. \\
	Значи, най-добре е да застанем на двадесетото място в редицата.
	

	\subsection*{Задача 11:}
	Нека $X_1, X_2, .., X_n \sim U([0,1])$ са независими и еднакво разпределени случайни величини. Намерете $\probP(X_1 + X_2 + ... + X_n \le 1)$.
	
	\subsection*{Решение:}	
	Наредената $n$-торка от тези величини е случайно избрана точка от единичния n-мерен куб. $X_1 + X_2 + ... + X_n = 1$ е уравнение на хиперравнина, която минава през върховете от вида $(0, 0, ..., 0, 1, 0, ..., 0)$. Тези върхове са съседите на върха $(0, 0, ..., 0)$. \\
	Неравенството $(X_1 + X_2 + ... + X_n \le 1)$ задава полупространство, и то частта от хиперкуба, която лежи към върха $(0, 0, ..., 0)$.
	Това е $n$-мерна пирамида. \\ \\
	Понеже разпределението е равномерно, то търсената вероятност е равна на
	обема на пирамидата върху обема на целия куб $= \dfrac{1/n!}{1} = \dfrac{1}{n!}$. \\
	\\
	\\
	Обемът $\dfrac{1}{n!}$ на $n$-мерна пирамида с прав многостенен ъгъл и единични ръбове на този ъгъл се доказва по индукция: \\
	Лицето на основата е $\dfrac{1}{(n-1)!}$ според индуктивното предположение. От свойствата на подобието следва, че $"$триъгълникът$"$ на разстояние $x$ от върха на пирамидата има лице $\dfrac{x^{n-1}}{(n-1)!}$. Интегрираме това лице по $x$ от $0$ до $1$ и получаваме $\dfrac{1}{n!}$.
	Това е индуктивната стъпка, а базата е $n = 1$ (дължина на отсечка).

\end{document}